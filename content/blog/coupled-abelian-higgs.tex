The Abelian Higgs model is perhaps the simplest example of spontaneous symmetry
breaking which gives mass to the boson.  The Lagrangian is given by

\begin{equation}
  \mathscr{L} = - \frac{1}{4} F_{\mu\nu}^{2} + \abs{\partial_{\mu} \phi + i e A_{\mu} \phi}^{2} + m^{2} \abs{\phi}^2 - \lambda \abs{\phi}^{4};
\end{equation}

however, I am interested in the next simplest scenario which involves two Higgs
charged under the same gauge symmetry.  The Lagrangian becomes:

\begin{align}
\mathscr{L} &= - \frac{1}{4} F_{\mu\nu}^{2} + \abs{\partial_{\mu} \phi_{1} + i e_{2} A_{\mu} \phi}^{2} + m_{1}^{2} \abs{\phi_{1}}^2 - \lambda_{1} \abs{\phi_{1}}^{4} \\
            &\quad + \abs{\partial_{\mu} \phi_{2} + i e_{2} A_{\mu} \phi}^{2} + m_{2}^{2} \abs{\phi_{2}}^2 - \lambda_{2} \abs{\phi_{2}}^{4} + \mathscr{L}_{\text{int}}
\end{align}